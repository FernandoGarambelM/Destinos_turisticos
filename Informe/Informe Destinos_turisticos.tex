%package list
\documentclass{article}
\usepackage[top=3cm, bottom=3cm, outer=3cm, inner=3cm]{geometry}
\usepackage{multicol}
\usepackage{graphicx}
\usepackage{url}
%\usepackage{cite}
\usepackage{hyperref}
\usepackage{array}
%\usepackage{multicol}
\newcolumntype{x}[1]{>{\centering\arraybackslash\hspace{0pt}}p{#1}}
\usepackage{natbib}
\usepackage{pdfpages}
\usepackage{multirow}
\usepackage[normalem]{ulem}
\useunder{\uline}{\ul}{}
\usepackage{svg}
\usepackage{xcolor}
\usepackage{listings}
\lstdefinestyle{ascii-tree}{
    literate={├}{|}1 {─}{--}1 {└}{+}1 
  }
\lstset{basicstyle=\ttfamily,
  showstringspaces=false,
  commentstyle=\color{red},
  keywordstyle=\color{blue}
}
%\usepackage{booktabs}
\usepackage{caption}
\usepackage{subcaption}
\usepackage{float}
\usepackage{array}

\newcolumntype{M}[1]{>{\centering\arraybackslash}m{#1}}
\newcolumntype{N}{@{}m{0pt}@{}}


%%%%%%%%%%%%%%%%%%%%%%%%%%%%%%%%%%%%%%%%%%%%%%%%%%%%%%%%%%%%%%%%%%%%%%%%%%%%
%%%%%%%%%%%%%%%%%%%%%%%%%%%%%%%%%%%%%%%%%%%%%%%%%%%%%%%%%%%%%%%%%%%%%%%%%%%%
\newcommand{\itemEmail}{fgarambel@unsa.edu.pe}
\newcommand{\itemStudent}{Fernando Miguel Garambel Marín}
\newcommand{\itemCourse}{Laboratorio de Programación Web 2}
\newcommand{\itemCourseCode}{1701212}
\newcommand{\itemSemester}{III}
\newcommand{\itemUniversity}{Universidad Nacional de San Agustín de Arequipa}
\newcommand{\itemFaculty}{Facultad de Ingeniería de Producción y Servicios}
\newcommand{\itemDepartment}{Departamento Académico de Ingeniería de Sistemas e Informática}
\newcommand{\itemSchool}{Escuela Profesional de Ingeniería de Sistemas}
\newcommand{\itemAcademic}{2024 - A}
\newcommand{\itemInput}{Del 24 de mayo 2024}
\newcommand{\itemOutput}{Al 5 de junio 2024}
\newcommand{\itemPracticeNumber}{06}
\newcommand{\itemTheme}{Django - Usando una plantilla para ver Destinos Turísticos}
%%%%%%%%%%%%%%%%%%%%%%%%%%%%%%%%%%%%%%%%%%%%%%%%%%%%%%%%%%%%%%%%%%%%%%%%%%%%
%%%%%%%%%%%%%%%%%%%%%%%%%%%%%%%%%%%%%%%%%%%%%%%%%%%%%%%%%%%%%%%%%%%%%%%%%%%%

\usepackage[english,spanish]{babel}
\usepackage[utf8]{inputenc}
\AtBeginDocument{\selectlanguage{spanish}}
\renewcommand{\figurename}{Figura}
\renewcommand{\refname}{Referencias}
\renewcommand{\tablename}{Tabla} %esto no funciona cuando se usa babel
\AtBeginDocument{%
	\renewcommand\tablename{Tabla}
}

\usepackage{fancyhdr}
\pagestyle{fancy}
\fancyhf{}
\setlength{\headheight}{30pt}
\renewcommand{\headrulewidth}{1pt}
\renewcommand{\footrulewidth}{1pt}
\fancyhead[L]{\raisebox{-0.2\height}{\includegraphics[width=3cm]{img/logo_episunsa.png}}}
\fancyhead[C]{\fontsize{7}{7}\selectfont	\itemUniversity \\ \itemFaculty \\ \itemDepartment \\ \itemSchool \\ \textbf{\itemCourse}}
\fancyhead[R]{\raisebox{-0.2\height}{\includegraphics[width=1.2cm]{img/logo_abet}}}
\fancyfoot[L]{Fernando Garambel}
\fancyfoot[C]{\itemCourse}
\fancyfoot[R]{Página \thepage}

% para el codigo fuente
\usepackage{listings}
\usepackage{color, colortbl}
\definecolor{dkgreen}{rgb}{0,0.6,0}
\definecolor{gray}{rgb}{0.5,0.5,0.5}
\definecolor{mauve}{rgb}{0.58,0,0.82}
\definecolor{codebackground}{rgb}{0.95, 0.95, 0.92}
\definecolor{tablebackground}{rgb}{0.8, 0, 0}

\lstset{frame=tb,
	language=bash,
	aboveskip=3mm,
	belowskip=3mm,
	showstringspaces=false,
	columns=flexible,
	basicstyle={\small\ttfamily},
	numbers=none,
	numberstyle=\tiny\color{gray},
	keywordstyle=\color{blue},
	commentstyle=\color{dkgreen},
	stringstyle=\color{mauve},
	breaklines=true,
	breakatwhitespace=true,
	tabsize=3,
	backgroundcolor= \color{codebackground},
}

\begin{document}
	
	\vspace*{10px}
	
	\begin{center}	
		\fontsize{17}{17} \textbf{ Informe de Laboratorio \itemPracticeNumber}
	\end{center}
	\centerline{\textbf{\Large Tema: \itemTheme}}
	%\vspace*{0.5cm}	

	\begin{flushright}
		\begin{tabular}{|M{2.5cm}|N|}
			\hline 
			\rowcolor{tablebackground}
			\color{white} \textbf{Nota}  \\
			\hline 
			     \\[30pt]
			\hline 			
		\end{tabular}
	\end{flushright}	

	\begin{table}[H]
		\begin{tabular}{|x{4.7cm}|x{4.8cm}|x{4.8cm}|}
			\hline 
			\rowcolor{tablebackground}
			\color{white} \textbf{Estudiante} & \color{white}\textbf{Escuela}  & \color{white}\textbf{Asignatura}   \\
			\hline 
			{\itemStudent \par \itemEmail} & \itemSchool & {\itemCourse \par Semestre: \itemSemester \par Código: \itemCourseCode}     \\
			\hline 			
		\end{tabular}
	\end{table}		
	
	\begin{table}[H]
		\begin{tabular}{|x{4.7cm}|x{4.8cm}|x{4.8cm}|}
			\hline 
			\rowcolor{tablebackground}
			\color{white}\textbf{Laboratorio} & \color{white}\textbf{Tema}  & \color{white}\textbf{Duración}   \\
			\hline 
			\itemPracticeNumber & \itemTheme & 04 horas   \\
			\hline 
		\end{tabular}
	\end{table}
	
	\begin{table}[H]
		\begin{tabular}{|x{4.7cm}|x{4.8cm}|x{4.8cm}|}
			\hline 
			\rowcolor{tablebackground}
			\color{white}\textbf{Semestre académico} & \color{white}\textbf{Fecha de inicio}  & \color{white}\textbf{Fecha de entrega}   \\
			\hline 
			\itemAcademic & \itemInput &  \itemOutput  \\
			\hline 
		\end{tabular}
	\end{table}

\section{Objetivos}
	\begin{itemize}		
		\item Implementar una aplicación en Django utilizando una plantilla profesional.
		\item Utilizar una tabla de Destinos turísticos para leer y completar la página web
		\item Utilizar los tags “if” y “for” en los archivos html para leer todos los registros de una tabla desde una base de datos.

	\end{itemize}
\section{Actividades}
	\begin{itemize}		
		\item Crear un proyecto en Django
		\item Siga los pasos del video para poder implementar la aplicación de Destinos turísticos
		\item Use git y haga los commits necesarios para manejar correctamente la aplicación.
	\end{itemize}
\section{Ejercicio Propuestos}
	\begin{itemize}		
		\item Deberán replicar la actividad del video que se encuentra en el AV de Teoría (Django Tutorial for Beginners - Telusko( \url{https://youtu.be/OTmQOjsl0eg} )donde se obtiene una plantilla de una aplicación de Destinos turísticos y adecuarla a un proyecto en blanco Django.
		\item Luego trabajar con un modelo de tabla DestinosTuristicos donde se guarden nombreCiudad,  descripcionCiudad, imagenCiudad, precioTour, ofertaTour (booleano).  Estos destinos turísticos deberán ser agregados en una vista dinámica utilizando tags for e if.
		\item Para ello crear una carpeta dentro del proyecto github colaborativo con el docente, e informar el link donde se encuentra.
		\item Crear formularios de Añadir Destinos Turísticos, Modificar, Listar y Eliminar Destinos.  
	\end{itemize}
\section{Equipos, materiales y temas utilizados}
	\begin{itemize}
		\item Sistema operativo de 64 bits, procesador basado en x64.
		\item Latex. 
		\item git version 2.41.0.windows.1
		\item Cuenta en GitHub con el correo institucional.
	\end{itemize}
\section{URL Github, Video}
	\begin{itemize}
		\item URL del Repositorio GitHub para clonar o recuperar.
		\item \url{https://github.com/FernandoGarambelM/Destinos_turisticos.git}
		\item URL para el video flipgrid
		\item \url{Video}	
	\end{itemize}
	\clearpage
\section{Replicando la actividad del video}
	\begin{itemize}
		\item Primero ingresamos al entorno virtual
	\end{itemize}
	\begin{lstlisting}[language=bash,caption={Activar el ambiente virtual}][H]
		env\Scripts\activate
	\end{lstlisting}
	\begin{lstlisting}[language=bash,caption={Clonar el repositorio}][H]
		git clone https://github.com/navinreddy20/django-telusko-codes
	\end{lstlisting}
	\begin{itemize}
		\item Luego instalamos postgresql y pgadmin 
		\item Luego instalamos al entorno virtual psycopg2-binary y Pillow
	\end{itemize}
	\begin{lstlisting}[language=bash,caption={Instalando psycopg2-binary y Pillow}][H]
		pip install psycopg2-binary
		pip install Pillow
	\end{lstlisting}
	\begin{itemize}
		\item  Despues de instalar, creamos un superusuario con el siguiente script
	\end{itemize}
\begin{lstlisting}[language=bash,caption={Script para crear un superusuario}][H]

#!/bin/sh

# Variables de entorno para el superusuario
DJANGO_SUPERUSER_USERNAME=admin
DJANGO_SUPERUSER_EMAIL=admin@example.com
DJANGO_SUPERUSER_PASSWORD=123456

# Ejecutar el comando createsuperuser sin interaccion
python manage.py shell -c "from django.contrib.auth import get_user_model; User = get_user_model(); User.objects.create_superuser('$DJANGO_SUPERUSER_USERNAME', '$DJANGO_SUPERUSER_EMAIL', '$DJANGO_SUPERUSER_PASSWORD') if not User.objects.filter(username='$DJANGO_SUPERUSER_USERNAME').exists() else print('Superusuario ya existe.')"
	\end{lstlisting}
	\begin{itemize}
		\item Realizamos migraciones para enlazar la base de datos
	\end{itemize}
	\begin{lstlisting}[language=bash,caption={Codigo para realizar migraciones}][H]
		python manage.py makemigrations
		python manage.py migrations
	\end{lstlisting}
	\begin{itemize}
		\item Despues de seguir estos pasos la plantilla ya funciona por completo
	\end{itemize}
\section{Adecuar la plantilla a un proyecto en blanco de Django}
	\begin{itemize}
		\item Creamos el modelo destination con los datos requeridos
	\end{itemize}
\lstinputlisting[language=Python, caption={Código de models.py},numbers=left,]{src/models1.py}
	\begin{itemize}
		\item Luego creamos los forms 
	\end{itemize}
\lstinputlisting[language=Python, caption={Código de forms.py},numbers=left,]{src/forms.py}
	\begin{itemize}
		\item Luego creamos views.py
	\end{itemize}	
\lstinputlisting[language=Python, caption={Código de views.py},numbers=left,]{src/views.py}
	\begin{itemize}
		\item Luego agregamos urls.py para que Django pueda leerlos
	\end{itemize}	
\lstinputlisting[language=Python, caption={Código de urls.py},numbers=left,]{src/urls.py}	
	\begin{itemize}
		\item Usamos la  plantilla de travello, especificamente la siguiente parte para listar los destinos turisticos con if y for
	\end{itemize}	
	\begin{lstlisting}[language=bash,caption={Pedazo de html donde se usa  for e if}][H]
						
							<!-- Destination -->
							<div class="destination item">
								<div class="destination_image">
									<img src="{{dest.img.url}}" alt="">

									
									<div class="spec_offer text-center"><a href="#">Special Offer</a></div>
									
								</div>
								<div class="destination_content">
									<div class="destination_title"><a href="destinations.html">{{dest.name}}</a>
									</div>
									<div class="destination_subtitle">
										<p>{{dest.desc}}</p>
									</div>
									<div class="destination_price">From ${{dest.price}}</div>
								</div>
							</div>

							

	\end{lstlisting}
	\begin{figure}[H]
		\centering
		\includegraphics[width=1.0\textwidth,keepaspectratio]{img/Vista3.png}
		%\includesvg{img/automata.svg}
		%\label{img:mot2}
		%\caption{Product backlog.}
	\end{figure}
\section{Referencias}
\begin{itemize}			
	\item \url{https://docs.djangoproject.com/es/3.2/}
	\item\url{https://docs.djangoproject.com/es/3.2/ref/models/fields/#field-types}
\end{itemize}	
	
%\clearpage
%\bibliographystyle{apalike}
%\bibliographystyle{IEEEtranN}
%\bibliography{bibliography}
			
\end{document}